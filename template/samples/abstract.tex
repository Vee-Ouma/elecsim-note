Due to the threat of climate change, a shift from a fossil-fuel based system to one based on zero-carbon is required. However, this is not as simple as instantaneously closing down all fossil fuel energy generation and replacing them with renewable sources -- careful decisions need to be taken. To aid decision makers, we present a new tool, ElecSIM, which is an agent-based modelling framework used to examine the effect of policy on long term investment decisions. We review different techniques currently used to model long term energy decisions, and motivate why agent-based models will become an important strategic tool for policy makers.

We show that modelling stochasticity improves model reliability by $52.5\%$, and motivate why an open-source toolkit is required. We demonstrate how ElecSIM meets the requirements of the electricity market. The model runs in yearly time steps, making assumptions based on empirical data on the impact of intermittent renewable energy. We present the dynamics of the system through scenario testing and provide validation. ElecSIM allows non-experts to rapidly prototype new ideas, and is developed around a modular framework -- which allows technical experts to add and remove features at will. 


We demonstrate the importance of a carbon tax and making the transition a low-carbon electricity supply. With a value of \textsterling70 to achieve close to 100\% renewable energy. An interesting note is that starting with a low carbon tax and slowly increasing this to the year 2050 provides minimal benefits on projections to 2050. A slightly more aggressive short term tax, whilst maintaining this level has similar effects as starting low and finishing high. This has the benefits of reducing costs as well as providing certainty to investors.
