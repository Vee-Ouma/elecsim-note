Due to the threat of climate change, a transition from a fossil-fuel based system to one based on zero-carbon is required. However, this is not as simple as instantaneously closing down all fossil fuel energy generation and replacing them with renewable sources -- careful decisions need to be taken to ensure rapid but stable progress. To aid decision makers, we present a new tool, ElecSIM, which is an open-sourced agent-based modelling framework used to examine the effect of policy on long term investment decisions in the electricity sector. ElecSIM allows non-experts to rapidly prototype new ideas, and is developed around a modular framework -- which allows technical experts to add and remove features at will. 


%We show that modelling stochasticity of the system improves model reliability by $52.5\%$. We further provide motivational arguments as to why an open-source toolkit is required. We demonstrate how ElecSIM meets the requirements of the electricity market. The model runs in yearly time steps, making assumptions based on empirical data on the impact of intermittent renewable energy and historical generation prices. We present the dynamics of the system through scenario testing and provide validation of parametrisation. ElecSIM allows non-experts to rapidly prototype new ideas, and is developed around a modular framework -- which allows technical experts to add and remove features at will. 

We review different techniques currently used to model long term electricity decisions, and use this to motivate why agent-based models will become an important strategic tool for policy makers. We provide motivational arguments as to why an open-source toolkit is required to model long-term electricity markets.

We compare actual electricity prices with our model and demonstrate that the modelling of stochasticity in the system improves performance by $52.5\%$

We demonstrate how effective a carbon tax is at encouraging a low-carbon electricity supply market and show how a \textsterling70 ($\$90$) per tonne of carbon emitted would lead to an almost 100\% renewable electricity energy market by 2050. An interesting note, however, is that starting with a low carbon tax and slowly increasing this by the year 2050 provides similar benefits to a lower, but consistent tax in the long run, due to the high capital costs and long operating periods of generators. This has the benefits of reducing costs as well as providing certainty to investors.
