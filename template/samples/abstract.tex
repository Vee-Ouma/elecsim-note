Due to the threat of climate change, a shift from a fossil-fuel based system to one based on zero-carbon is required. However, this is not as simple as instantaneously closing down all fossil fuel energy generation and replacing them with renewable sources -- careful decisions need to be taken. To aid decision makers, we present a new tool, ElecSIM, which is an agent-based modelling framework used to examine the effect of policy on long term investment decisions. We review different techniques currently used to model long term energy decisions, and motivate why agent-based models will become an important strategic tool for policy makers.

We show that modelling stochasticity improves model reliability by $52.5\%$, and motivate why an open-source toolkit is required. We demonstrate how ElecSIM meets the requirements of the electricity market. The model runs in yearly time steps, making assumptions based on empirical data on the impact of intermittent renewable energy. We present the dynamics of the system through scenario testing and provide validation. ElecSIM allows non-experts to rapidly prototype new ideas, and is developed around a modular framework -- which allows technical experts to add and remove features at will. 

{\color{red}
This bit seems to be saying "and our work is not complete yet" -- better to say something about the results here (when we have them) and leave the text below for the conclusions.

Future work includes integrating different types of agent based learning for the bidding and investment process, utilising multi-agent reinforcement algorithms that can deal with a non stationary environment. We will use the yearly time-step as a baseline model for integration of a higher temporal and spatial resolution.
}