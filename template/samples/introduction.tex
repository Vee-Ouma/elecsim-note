
The world faces significant challenges from climate change and global warming {\color{red}[CITE]}. A rise in carbon emissions increases the risk of severe impacts on the world such as rising sea levels, species extinction, heat waves and tropical cyclones \cite{IPCC2014}. The scientific literature concurs that the recent change in climate is anthropogenic, with 97\% of peer reviewed articles of this view \cite{Cook2013}.  

To achieve carbon neutrality, the energy mix must shift from a largely fossil fuel based system, to one based on renewable energy. In essence, using solar, wind and tidal power to generate electricity and power homes, industry and transport \cite{Hoffert2002}. Electricity is a significant proprtion of our energy consumption -- consuming {\color{red}XX\%} of energy usage per year {\color{red}[CITE]}. Although other forms of energy consumption are important we focus here only on the production and consumption of electricity. {\color{red}I'm wondering if some sort of graph / table of different electricity generation techniques, how common they are and how carbon neutral they are would go well in here?}

For this to occur, a transition in electricity infrastructure is required. Moving from a centralised and homogenous fossil fuel-based system to a distributed system based on renewable energy and batteries. However, such a transition needs to be performed in a safe and non-disruptive manner -- it may be possible to close down all fossil fuel plants in the next year, though if this leads to electricity shortages and power cuts then this is likely to cause significant problems both for companies and homes. Therefore a stepped approach which allows seamless transfer is desirable. This may seem a simple process to achieve -- slowly phase out existing fossil fuel generators and replace by renewable sources -- however, there are many risks and uncertainties in this process. Existing power plants have an expected lifetime and their owners wish to maximise this and the profits which can be made from them, renewable sources are still developing meaning that their efficiency and reliability will change in years to come, along with the fact that most renewable sources are effected by conditions outside the control of the owners (e.g. time or day or wind speed) thus leading to a need for electricity storage. {\color{red}I'm sure there are better / more.}

To better understand the risks and uncertainties surrounding this transition, and to model the potential actions that can be taken by policy makers, this paper presents ElecSIM, an open source agent-based modelling toolkit, written in Python, which allows for the evaluation of alternative scenarios prior to implementation of policy. Through simulation we can evaluate many strategies in order to identify those most likely to achieve our requirements of rapid but non-disruptive migration from fossil to renewable.

This tool can be used by:
\begin{itemize}
	\item {\bf Policy experts} to test policy outcomes under different scenarios and provide quantitative advice to policy makers. They can provide a simple script defining the policies they wish to use along with the parameters for these polices.
	\item {\bf Energy market developers} who can use the extensible framework to add such things as new energy sources, policy types, consumer profiles and storage types. Thus allowing ElecSIM to adapt to a changing ecosystem.
\end{itemize}
International agreements such as the Paris climate agreement {\color{red}[CITE]}, where nation states agreed on the goal of limiting the rise in global average temperature to well below 2$^\circ$C above pre-industrial levels, mean that an open-source, reproducible and transparent model that can be taken to account by experts and understood by non-experts is of importance. This allows for the development of policies based on known assumptions, thorough testing and validation.

{\color{green} Not quite clear --
Optimisation models calculate an optimal cost pathway of investment in power plants over the long-term. However, many western democracies have purposely moved away from central control in the energy sector, which make the results of optimisation models difficult to implement. Agent-based models provide a solution to this by modelling heterogeneous actors with imperfect knowledge in a regulatory environment. The emergent behaviour of the agents can be observed under different policies such as a carbon tax.}

Policy making in energy system comes with inherent risk. Decisions made can have large long term impacts and may be sub-optimal. Power plants have high capital costs, long construction times, and operate over a long period. Therefore, errors made may be compounded, and can have effects well into the future. {\color{red} Think this is quite crucial -- though needs drawing out more clearly -- there is no overseer in the energy industry and each player can act independently. Hence we can influence the different players but not force them to do something. A diagram showing the different players, who can influence them and how?}

This paper details our model, ElecSIM. Section \ref{Literature Review} is a literature review of the models currently used in practice. Section \ref{Model} details the model and assumptions made, and section \ref{Valdiation and Performance} details how we validated our model, and displays performance metrics. Section \ref{Scenario Testing} details our results, and explores ways in which ElecSIM can be used. We conclude the work in section \ref{Conclusion}
 

\begin{itemize}
	\item We have developed a framework for evaluating alternative scenarios, prior to implementation of policy.
	\item Used by experts working in collaboration with policy makers.
	\item Importance of a transition in electricity infrastructure (Paris agreement, UK Climate change act)
	\item Importance of understanding effect of decisions made today on the future (limit of 1.5C by 2050)
	\item Introduce ElecSIM as a toolkit to inform long-term domestic policy questions in the electricity market. 
	\item Ability to model the effects of carbon taxation, and the effect of different scenarios 
	\item Talk about the need to model a non-stationary, dynamic system, with multiple interacting agents with imperfect information
	\item Requirement for an Open-Source, free Toolkit written in python. Low barrier of entry, and integration with existing python data analytics and machine learning techniques. Transparent, reproducible, and data made available. This allows for results to be open to greater criticism and better inform policy decisions.
	\item Simple model which matches real life behaviour for low complexity and therefore increases transparency.
\end{itemize}

