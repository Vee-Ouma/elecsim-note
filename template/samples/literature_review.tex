
Energy models can typically be classified as bottom-up techno-economic models or top-down macro-economic models \cite{Bohringer1998}. Top-down models model how demand varies with regards to historical economic data, and analyse aggregate behaviour \cite{Hall2016}. They model how energy prices vary with respect to elasticities. Bottom-up models represent the energy sector in detail, and are written as mathematical programming problems \cite{Gargiulo2013}. They detail technology explicitly, and can include cost and emissions implications \cite{Hall2016}.

It is possible to further categorise energy models into optimisation and simulation models. Optimisation energy models minimise costs or maximise welfare from the perspective of a central planner, for instance a government \cite{Keles2017}. A use-case would be a government that wants cheap, reliable and a low-carbon electricity supply by a future date. An optimisation model would find the optimal mix of generators to meet this whilst taking into account the constraints. 

However, electricity market liberalisation in many Western democracies has changed the framework conditions. A central planner has given way to multiple heterogeneous agents acting in their best interest \cite{Most2010}. Therefore, certain policy options must be used by a central planner to attain a desired outcome, for example carbon taxes or subsidies. It is therefore proposed that these complex agents are modelled using agent-based modelling.

As a result of this, agent-based simulation has received increasing attention in recent years, and a number of simulation tools have emerged, for example SEPIA\cite{Kraan2018} EMCAS \cite{Conzelmann}, NEMSIM, AMES, PowerACE, MACSEM, and GAPEX

SEPIA is a discrete event agent based model which utilises agent learning for agent behaviour \cite{Harp2000}. However, the model is utilised for industry specific questions such as .

\begin{itemize}
	\item Agent Based Models - eg. EMCAS, PowerACE, EMLab: Leaves a requirement for an open source toolkit written in python. Many one-off models available, however difficult to apply to different scenarios.
	(SEPIA [6], EMCAS [7], NEMSIM [8], AMES [9], PowerACE [10], MASCEM [11, 12], and GAPEX [13] \cite{Lopes})
	\item Bottom-up optimization models to find minimum cost of electricity system. \cite{Pfenninger2014}. eg. MARKAL/TIMES, MESSAGE. (These do not provide information on how to achieve a certain goal, particularly in a liberalized energy market. Or scenarios as to why a goal may not be achieved as the goal is assumed to be achieved.)
	\item Computational general equilibrium (CGE) models - Top-down macroeconomic models partial equilibrium model (energy supply, demand, cross-border trade, emissions)- Can be highly complex and difficult to understand. eg. NEMS, PRIMES.
\end{itemize}



