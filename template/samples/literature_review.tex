Live experimentation of physical processes is often not practical. The costs of real life experimentation can be prohibitively high, and it normally requires significant time in order to fully ascertain the long-term trends. There is also a risk that changes can have detrimental impacts ~\cite{Forshaw2016}. These factors are particularly true for an electricity market, where decisions made can have long term impacts on energy mix, carbon emissions and agent behaviour.  A solution to this is simulation, which can be used for rapid testing and prototyping of ideas. Simulation is the substitution of a physical process with a computer model. The computer model is parametrised by real world data and phenomena. The user is then able to experiment using this model, and assess the likelihoods of outcomes under certain scenarios and input variables \cite{Law:603360}.

\begin{table*}[]
	\begin{tabular}{|l|c|c|c|c|c|}
		\hline
		\multicolumn{1}{|c|}{\textbf{Tool name}} & \textbf{Open Source} & \textbf{Long-Term Investment} & \textbf{Market} & \textbf{Stochastic Inputs} & \textbf{Country Generalisability} \\ \hline
		SEPIA                                    & \checkmark           & x                             & \checkmark      & Demand                     & \checkmark                        \\ \hline
		EMCAS                                    & x                    & \checkmark                    & \checkmark      & Outages                    & \checkmark                        \\ \hline
		NEMSIM                                   & Unknown              & \checkmark                    & \checkmark      & x                          & x                                 \\ \hline
		AMES                                     & \checkmark           & x                             & Day-ahead       & x                          & x                                 \\ \hline
		PowerACE                                 & x                    & \checkmark                    & \checkmark      & Outages/Demand             & \checkmark                        \\ \hline
		MACSEM                                   & Unknown              & x                             & \checkmark      & x                          & \checkmark                        \\ \hline
		GAPEX                                    & Unknown              & x                             & Day-ahead       & x                          & \checkmark                        \\ \hline
		EMLab                                    & \checkmark           & \checkmark                    & Futures         & x                          & \checkmark                        \\ \hline
		ElecSIM                                  & \checkmark           & \checkmark                    & Futures         & \checkmark                 & \checkmark                        \\ \hline
	\end{tabular}
\end{table*}


Energy policy modelling is an example where simulation can be used. Real-life experimentation of energy policy is not always feasible, and as discussed, decisions can have long-term impacts. A number of different simulations and computer models have been used to aid policy makers and energy market developers in coming to informed conclusions.

Energy models can typically be classified as top-down macro-economic models or bottom-up techno-economic models~\cite{Bohringer1998}. Top-down models generally focus on behavioural realism with a focus on macro-economic metrics. They are useful for studying economy-wide responses to policies. ~\cite{Hall2016}, for example MARKAL-MACRO \cite{Fishbone1981} and LEAP \cite{Heaps2016}. Bottom-up models represent the energy sector in detail, and are written as mathematical programming problems~\cite{Gargiulo2013}. They detail technology explicitly, and can include cost and emissions implications~\cite{Hall2016}.

It is possible to further categorise bottom-up models into optimisation and simulation models. Optimisation energy models minimise costs or maximise welfare from the perspective of a central planner, for instance a government~\cite{Keles2017}. A use-case would be a government that wants cheap, reliable and low-carbon electricity supply by a future date. An optimisation model would find the optimal mix of generators to meet this whilst taking into account the constraints. Examples of optimisation models are MARKAL/TIMES~\cite{Fishbone1981} and MESSAGE~\cite{Schrattenholzer1981}. MARKAL is possibly the most widely used general purpose energy systems model~\cite{Pfenninger2014}.

However, electricity market liberalisation in many Western democracies has changed the framework conditions. Centralised, monopolistic, decision making entities has given way to multiple heterogeneous agents acting in their own best interest~\cite{Most2010}. Therefore, certain policy options which encourage changes must be used by a central planner to attain a desired outcome, for example carbon taxes or subsidies. It is therefore proposed that these complex agents are modelled using agent-based modelling.

As a result of this, agent-based simulation has received increasing attention in recent years, and a number of simulation tools have emerged, for example SEPIA~\cite{Harp2000} EMCAS~\cite{Conzelmann}, NEMSIM~\cite{Batten2006}, AMES~\cite{Sun2007}, PowerACE~\cite{Rothengatter2007}, ~\cite{Praca2003}, GAPEX~\cite{Cincotti2013} and  EMLab~\cite{Chappin2017}. However, none of which suit the needs of an open source, long-term market model which has a stochastic representation of input variables.

SEPIA \cite{Harp2000} is a discrete event agent based model which utilises Q-learning for agent behaviour. SEPIA models plants as being always on, and does not have an independent system operator (ISO), which in an electricity market, is an independent non-profit organization for coordinating and controlling of regular operations of the electric power system and market  \cite{Zhou2007}. SEPIA does not model a spot market, instead focusing on bilateral contracts. As opposed to this, ElecSIM has been designed with a merit-order, spot market in mind and renewable energy supply running intermittently.

MACSEM \cite{Praca2003} simulates a bilateral and pool market. It has been used to probe the effects of market rules and conditions by simulating and testing different bidding strategies. However, MACSEM does not model long term investment decisions.

EMCAS ~\cite{Conzelmann} is a closed source agent-based framework which investigates the interactions between physical infrastructures and economic behaviour of market participants. ElecSIM, however, focuses on purely the dynamics on the market, with an aim of providing a simplified, transparent, open source model of market operation, whilst maintaining robustness.

PowerACE ~\cite{Rothengatter2007} is also a closed source agent-based simulation of electricity markets that integrates short-term perspectives of daily electricity trading and long-term investment decisions. Similarly to ElecSIM, PowerACE initialises agents with all power plants in their respective country. However, unlike ElecSIM, PowerACE does not take into account stochasticity of price risks in electricity markets which is of crucial importance to real markets~\cite{Most2010}.

EMLab ~\cite{Chappin2017} is also an agent-based modelling toolkit for the electricity market. EMLab models an endogenous European emissions trading scheme with a yearly time-step. However, like PowerACE, EMLab differs from ElecSIM by not taking into account stochasticity in the electricity markets, such as outages, differing fuel prices within a year period and stochasticity in power plant operating costs. However, after correspondence with the authors, we were unable to run EMLab.

AMES ~\cite{Sun2007} is an agent-based model specific to the US Wholesale Power Market Platform. GAPEX \cite{Cincotti2013} is an agent-based framework for modelling and simulating power exchanges in MATLAB . GAPEX utilises an enhanced version of the reinforcement technique Roth-Erev to consider the presence of affine total cost functions. However, neither of these model the long-term dynamics that ElecSIM is designed for.

We therefore propose ElecSIM to fill a gap of an open source, long-term stochastic investment, agent-based model. {\color{red}Table with ticks and crosses to show desired features and which ones have them?}

\begin{itemize}
	\item Agent Based Models - eg. EMCAS, PowerACE, EMLab: Leaves a requirement for an open source toolkit written in python. Many one-off models available, however difficult to apply to different scenarios.
	(SEPIA [6], EMCAS [7], NEMSIM [8], AMES [9], PowerACE [10], MASCEM [11, 12], and GAPEX [13] \cite{Lopes})
	\item Bottom-up optimization models to find minimum cost of electricity system. \cite{Pfenninger2014}. eg. MARKAL/TIMES, MESSAGE. (These do not provide information on how to achieve a certain goal, particularly in a liberalized energy market. Or scenarios as to why a goal may not be achieved as the goal is assumed to be achieved.)
	\item Computational general equilibrium (CGE) models - Top-down macroeconomic models partial equilibrium model (energy supply, demand, cross-border trade, emissions)- Can be highly complex and difficult to understand. eg. NEMS, PRIMES.
\end{itemize}



