
Energy models can typically be classified as bottom-up techno-economic models or top-down macro-economic models \cite{Bohringer1998}. Top-down models model how demand varies with regards to historical economic data, and analyse aggregate behaviour \cite{Hall2016}. They model how energy prices vary with respect to elasticities. Bottom-up models represent the energy sector in detail, and are written as mathematical programming problems \cite{Gargiulo2013}. They detail technology explicitly, and can include cost and emissions implications \cite{Hall2016}.

It is possible to further categorise energy models into optimisation and simulation models. Optimisation energy models minimise costs or maximise welfare from the perspective of a central planner, for instance a government \cite{Keles2017}. A use-case would be a government that wants cheap, reliable and a low-carbon electricity supply by a future date. An optimisation model would find the optimal mix of generators to meet this whilst taking into account the constraints. 

However, electricity market liberalisation in many Western democracies has changed the framework conditions. Centralised, monopolistic, decision making entities has given way to multiple heterogeneous agents acting in their best interest \cite{Most2010}. Therefore, certain policy options must be used by a central planner to attain a desired outcome, for example carbon taxes or subsidies. It is therefore proposed that these complex agents are modelled using agent-based modelling.

As a result of this, agent-based simulation has received increasing attention in recent years, and a number of simulation tools have emerged, for example SEPIA\cite{Kraan2018} EMCAS \cite{Conzelmann}, NEMSIM \cite{Batten2006}, AMES\cite{Sun2007}, PowerACE \cite{Rothengatter2007}, MACSEM, GAPEX  \cite{Cincotti2013}, EMLab\cite{Chappin2017}.

SEPIA is a discrete event agent based model which utilises Q-learning for agent behaviour \cite{Harp2000}. However, SEPIA is not primarily a market model, and focuses on transmission of electricity between GenCos (generator companies) and ConCo's (consumer companies). Therefore, they model plants as being always on, and not subject to market forces. As opposed to this, ElecSIM has been designed with a merit-order market in mind, with peaker power plants running at times of high demand, and renewable energy supply running intermittently.

EMCAS is an agent-based framework which investigates the interactions between physical infrastructures and economic behaviour of market participants \cite{Conzelmann}. ElecSIM, however, focuses on purely the dynamics on the market, with the aim to provide a simplified but robust model of market operation. 

PowerACE is a closed source agent-based simulation of electricity markets that integrates short-term perspectives of daily electricity trading and long-term investment decisions \cite{Rothengatter2007}. Similarly to ElecSIM, PowerACE initialises agents with all power plants in Germany. However, unlike ElecSIM, PowerACE does not take into account stochasticity of price risks in electricity markets which is of crucial importance to real markets \cite{Most2010}.

EMLab is also an agent-based modelling toolkit for the electricity market \cite{Chappin2017}. EMLab models an endogenous European emissions trading scheme with a yearly time-step. However, like PowerACE, EMLab differs from ElecSIM by not taking into account stochasticity in the electricity markets, such as outages, differing fuel prices within a year period and randomness in power plant operating costs.

AMES is specific to the US Wholesale Power Market Platform \cite{Sun2007}, and GAPEX, which utilises the reinforcement technique Roth-Erev, does not model long-term dynamics \cite{Cincotti2013}.

\begin{itemize}
	\item Agent Based Models - eg. EMCAS, PowerACE, EMLab: Leaves a requirement for an open source toolkit written in python. Many one-off models available, however difficult to apply to different scenarios.
	(SEPIA [6], EMCAS [7], NEMSIM [8], AMES [9], PowerACE [10], MASCEM [11, 12], and GAPEX [13] \cite{Lopes})
	\item Bottom-up optimization models to find minimum cost of electricity system. \cite{Pfenninger2014}. eg. MARKAL/TIMES, MESSAGE. (These do not provide information on how to achieve a certain goal, particularly in a liberalized energy market. Or scenarios as to why a goal may not be achieved as the goal is assumed to be achieved.)
	\item Computational general equilibrium (CGE) models - Top-down macroeconomic models partial equilibrium model (energy supply, demand, cross-border trade, emissions)- Can be highly complex and difficult to understand. eg. NEMS, PRIMES.
\end{itemize}



