
The shift in electricity markets from a centrally controlled monopoly, to a liberalised market with many heterogeneous players has increased the need for a new type of modelling. We motivate that agent-based models can be used as a solution to this, by modelling many actors with individual properties. 

Agent-based models are able to model imperfect information as well as heterogeneous actors. ElecSIM models imperfect information through forecasting of electricity demand and future fuel and electricity prices. This leads to agents taking risk on their investments, and more realistically model market conditions.

We demonstrated that increasing carbon tax can lead to a significant increase in investment of low-carbon technologies such as onshore wind. However, an interesting result, was that early decisions have a long impact on the future energy mix. The market can be significantly changed through investment decisions made many years previously. 

Our future work includes comparing agent-learning techniques, using multi-agent reinforcement learning algorithms and artificial intelligence to allow agents to learn in a non-static environment. We propose the integration of a higher temporal and spatial resolution to model changes in daily demand, as well as capacity factors by region, and transmission effects.

%\begin{itemize}
%	\item Requirement for agent based models based on imperfect information, liberalised energy markets
%	\item Requirement for low barriers to entry open source model.
%	\item Discuss results
%	\item Future work:
%	\begin{itemize}
%		\item Embedding multi-agent intelligence such as Genetic Algorithms,  Q-learning and dynamic reinforcement learning
%		\item Raise spatial and temporal resolution.
%	\end{itemize}
%\end{itemize}

\FloatBarrier